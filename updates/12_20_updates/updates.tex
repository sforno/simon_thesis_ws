\documentclass[11pt,a4paper]{article}
%\usepackage[toc,page]{appendix}
\usepackage{graphicx}
\usepackage[a4paper]{geometry}
\usepackage{xcolor}
\usepackage{fancyhdr}
\usepackage{float}
\usepackage{setspace}
\usepackage[absolute]{textpos}
\usepackage{epstopdf}
%\usepackage[]{mcode} 	% To include matlab code
\usepackage{capt-of}
\usepackage{enumerate}
\usepackage{lastpage}
\usepackage{booktabs}
\usepackage{longtable}
\usepackage{array}
\renewcommand{\arraystretch}{1.5}

\usepackage[english]{babel}
\usepackage[utf8]{inputenc}
\usepackage{amsmath}
\usepackage{amsfonts}
\usepackage{graphicx}
\usepackage[colorinlistoftodos]{todonotes}
\usepackage{algorithm}
\usepackage{algpseudocode}

\usepackage{amsmath}
\usepackage{algorithm}
%\usepackage[noend]{algpseudocode}
\makeatletter
\def\BState{\State\hskip-\ALG@thistlm}
\makeatother

\usepackage{amsmath}
\usepackage{amsfonts}
\usepackage{amssymb}
\usepackage{eurosym}

% Header
\setlength{\headheight}{30pt}
\newgeometry{top=2.5cm, bottom = 1.5cm, left=2cm, right=2cm}
\pagestyle{fancy} 
\lhead{\includegraphics[height=0.8cm]{figures/{tue_logo}.png}}
%\lfoot{Group 4 - ``CASE"-HENK}
\cfoot{~}
\rfoot{Page \thepage ~of \pageref{LastPage}}

\usepackage{cleveref}
% Change cleveref reference eq. to equation same for figure
\crefname{equation}{equation}{equations}
\crefname{figure}{figure}{figures}
\crefname{table}{table}{tables}

% Change Section numbering to Problem 1
%\renewcommand{\thesection}{Problem \arabic{section}.}

\begin{document}
%\begin{titlepage}
%\vspace*{100pt}
%\begin{figure}
%\centering
%\includegraphics[width=0.5\textwidth]{figures/TUelogozondertekst}
%\end{figure}
%\begin{center}
%{ \huge \bfseries 4AT100 Automotive Systems Engineering Project\\[0.4cm] }
%\textsc{\Large Concept Project Plan}\\[0.5cm]
%
%\end{center}
%
%\vfill
%
%\renewcommand{\arraystretch}{1}
%
%\begin{flushleft} \large
%\begin{tabular}{l}
%Project Coordinators:\\
%Dr.Ir. A. van de Mortel-Fronczak (Asia) \\
%Dr.Ir. I. Barosan (Ion) \\
%\end{tabular}
%\end{flushleft}
%
%\begin{flushleft} \large
%\begin{tabular}{l l l l}
%Tutor: & & & \\
%L. Kefalidis (Lazaros) & & & \\
%& & & \\
%Authors:\hspace{30mm} 	& \hspace{35mm}	& \hspace{55mm} 	    		& 			\\
%S. Forno (Simone) 		& ​0978942		& T. de Mor\'ee (Tim)			& 0944052 	\\
%R.M.A. Goris (Rob) 		& 0808822		& T.M.A. van de Wiel (Thijs)	​& 0824530 	\\
%B.S. Haarsma (Bouke) 	& 0751757​		& H. Wils (Hielke) 				& 0807014 	\\
%\end{tabular}
%\end{flushleft}
%
%\begin{flushleft} \large
%\begin{tabular}{l}
%MSc. Programme Automotive Technology \\
%Eindhoven University of Technology \\
%\end{tabular}
%\end{flushleft}
%
%\begin{flushleft} \large
%\begin{tabular}{l}
%\today \hspace{8.4cm} Group 4 ``CASE"-HENK \\
%\end{tabular}
%\end{flushleft}
%
%\renewcommand{\arraystretch}{1.5}
%
%\end{titlepage}

\newgeometry{top=2.5cm, bottom = 3cm, left=2cm, right=2cm}

%\newpage
%
%\setcounter{tocdepth}{2}
%
%\tableofcontents
%\newpage


%------------------------------------------------


\section{Inserting the Kinect camera into Husky}
Short description on how to modify the Gazebo files to put Sick Laser range finder or alternatively the Kinect camera. it is sufficient to modify the husky-fabrickhalle.launch file to change the robot configuration. The \textbf{urdf} and \textbf{gazebo} files in the husky kinetic devel should remain the same. \\




\section{Tracking AR tags with ar{\_}track{\_}alvar package} \label{sec:ar}

The file \textbf{ar{\_}tracker.launch} has been modified for the camera image topic to use sensor{\_}msg/Pointclouds instead of sensor{\_}msg/Image messages. The frame is /camera{\_}frame{\_}optical link. \\
Tried:
\begin{itemize}
\item Drive around to see if the robot could actually detect the tag.
\item Increase the marker scale in the file Fabrickhalle.world to be 1 (1 meter??): \textbf{Does not solve the issue}, the \textbf{ar{\_}tracking package} does not output the pose topic for the marker in the $ar{\_}pose{\_}marker$ topic.
\item Commenting out the \textbf{remap from} tags from the ar{\_}tracker.launch  file: not working
\item Looking into the published camera topics.The packege requires a subscription to two kinds of message: sensor{\_}msgs/CameraInfo and sensor{\_}msgs/Image
\begin{itemize}
\item /camera/depth/camera{\_}info is of type sensor{\_}msg/CameraInfo - not empty if rviz is launched
\item /camera/rgb/camera{\_}info is of type sensor{\_}msg/CameraInfo
\item /camera/depth/image{\_}raw or rgb/imageraw is of type sensor{\_}msgs/Image - not empty msg
\item /camera/depth/points is of type sensor{\_}msgs/PointCloud2 - not empty
\item Same holds for above for \textbf{rgb} instead
\item /camera{\_}image is of type sensor{\_}msgs/PointCloud2 - \textbf{emtpy}
\item /camera{\_}info is of type sensor{\_}msg/CameraInfo - \textbf{empty}
\end{itemize}
\end{itemize}

\section{Solved}
By driving to the other side of the tag the \textbf{ar alvar} is able to create a link from the ar tag, however this is very unstable, the detection presents jumps.

\section{Next steps}
\begin{itemize}
\item Find a way to make more reliable tags in Gazebo, making models in .world files
\item Put two tags and see how the Kinect is able to recognize separate tags, output separate frames. For this case we might want to use the \textbf{bundle launch file}, which is able to track poses of tag bundles.
\item Now as general, what we have from the \textbf{ar{\_}track{\_}alvar} is the link between the camera and the tags. The relation between tags and the Gazebo reference origin is known by the \textbf{world} file, all the tags frames should be published into a topic to be visible by the system.
\end{itemize}


\begin{figure}[H]
	\center
	\includegraphics[width=.5\textwidth]{figures/gazebo_reference.png}
\end{figure}

\end{document}


% == TABLE ==
%begin{table}[h!]
 % \centering
  %\caption{Caption for the table.}
 % \label{tab:table1}
 % \begin{tabular}{ccc}
 %   \toprule
  %  Some & actual & content\\
   % \midrule
   % prettifies & the & content\\
   % as & well & as\\
  %  using & the & booktabs package\\
  %   \bottomrule
  %\end{tabular}
%\end{table}


% === ALGORITHM == 

{\_}

\iffalse % multi-comment tool
\begin{algorithm}[!h]
   \caption{Kirsch, Rohig algorithm}
    \begin{algorithmic}[1]
    	\State $St-1 = St$
        \For{$i = 1$ to $N$} \Comment{With N the number of particles in the filter set by maxparticle parameter}
            \State $Spread $ $particles$ $in$ $the$ $anchorbox$ $with$ $equations$ $1)$ $and$ $2)$ $of$ $[3]$ \Comment{This step is called $Global$ $Localization$}
            
            \State $xt[n] = p(xt|xt-1,ut)$ \Comment{Motion update - sample the particles from the motion update of the robot and move forward to estimate the error model functions}
            
        	\State $wt[n] = p(dnanoLOC|si)*p(dlaser|si)$ \Comment{Measurement update - si are the particles set with i the i-th index}
        	\State $St = St + <xt,wt>$ \Comment{add the state and weight to the total state space}
        	
        	\State $Perform$ $resampling$
        \EndFor
    \State $Return$ $St$

\end{algorithmic}
\end{algorithm}
\fi


\iffalse

\begin{figure}[!htb]
    \centering
    \begin{minipage}{.5\textwidth}
        \centering
        \includegraphics[width=0.7\linewidth, height=0.2\textheight]{figures/amcl_param}
        \caption{The $amcl$ tunable parameters}
        \label{fig:amcl_param}
    \end{minipage}%
    \begin{minipage}{0.5\textwidth}
        \centering
        \includegraphics[width=0.7\linewidth, height=0.2\textheight]{figures/my_amcl_gmapping}
        \caption{Result of the Gmapping for the simple indoor environment}
        \label{fig:myamcl_map}
    \end{minipage}
 \end{figure}
 
 
 
\begin{figure}[!htb]
	\center
	\includegraphics[width=1\textwidth]{figures/active_localization_node.png}
	\caption{An example of an active localization node}
	\label{fig:active_locnode}
\end{figure}


% underscore symbol {\_}


\fi