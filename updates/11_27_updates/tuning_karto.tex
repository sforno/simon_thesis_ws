\documentclass[11pt,a4paper]{article}
%\usepackage[toc,page]{appendix}
\usepackage{graphicx}
\usepackage[a4paper]{geometry}
\usepackage{xcolor}
\usepackage{fancyhdr}
\usepackage{float}
\usepackage{setspace}
\usepackage[absolute]{textpos}
\usepackage{epstopdf}
%\usepackage[]{mcode} 	% To include matlab code
\usepackage{capt-of}
\usepackage{enumerate}
\usepackage{lastpage}
\usepackage{booktabs}
\usepackage{longtable}
\usepackage{array}
\renewcommand{\arraystretch}{1.5}

\usepackage[english]{babel}
\usepackage[utf8]{inputenc}
\usepackage{amsmath}
\usepackage{amsfonts}
\usepackage{graphicx}
\usepackage[colorinlistoftodos]{todonotes}
\usepackage{algorithm}
\usepackage{algpseudocode}

\usepackage{amsmath}
\usepackage{algorithm}
%\usepackage[noend]{algpseudocode}
\makeatletter
\def\BState{\State\hskip-\ALG@thistlm}
\makeatother

\usepackage{amsmath}
\usepackage{amsfonts}
\usepackage{amssymb}
\usepackage{eurosym}

% Header
\setlength{\headheight}{30pt}
\newgeometry{top=2.5cm, bottom = 1.5cm, left=2cm, right=2cm}
\pagestyle{fancy} 
\lhead{\includegraphics[height=0.8cm]{figures/{tue_logo}.png}}
%\lfoot{Group 4 - ``CASE"-HENK}
\cfoot{~}
\rfoot{Page \thepage ~of \pageref{LastPage}}

\usepackage{cleveref}
% Change cleveref reference eq. to equation same for figure
\crefname{equation}{equation}{equations}
\crefname{figure}{figure}{figures}
\crefname{table}{table}{tables}

% Change Section numbering to Problem 1
%\renewcommand{\thesection}{Problem \arabic{section}.}

\begin{document}
%\begin{titlepage}
%\vspace*{100pt}
%\begin{figure}
%\centering
%\includegraphics[width=0.5\textwidth]{figures/TUelogozondertekst}
%\end{figure}
%\begin{center}
%{ \huge \bfseries 4AT100 Automotive Systems Engineering Project\\[0.4cm] }
%\textsc{\Large Concept Project Plan}\\[0.5cm]
%
%\end{center}
%
%\vfill
%
%\renewcommand{\arraystretch}{1}
%
%\begin{flushleft} \large
%\begin{tabular}{l}
%Project Coordinators:\\
%Dr.Ir. A. van de Mortel-Fronczak (Asia) \\
%Dr.Ir. I. Barosan (Ion) \\
%\end{tabular}
%\end{flushleft}
%
%\begin{flushleft} \large
%\begin{tabular}{l l l l}
%Tutor: & & & \\
%L. Kefalidis (Lazaros) & & & \\
%& & & \\
%Authors:\hspace{30mm} 	& \hspace{35mm}	& \hspace{55mm} 	    		& 			\\
%S. Forno (Simone) 		& ​0978942		& T. de Mor\'ee (Tim)			& 0944052 	\\
%R.M.A. Goris (Rob) 		& 0808822		& T.M.A. van de Wiel (Thijs)	​& 0824530 	\\
%B.S. Haarsma (Bouke) 	& 0751757​		& H. Wils (Hielke) 				& 0807014 	\\
%\end{tabular}
%\end{flushleft}
%
%\begin{flushleft} \large
%\begin{tabular}{l}
%MSc. Programme Automotive Technology \\
%Eindhoven University of Technology \\
%\end{tabular}
%\end{flushleft}
%
%\begin{flushleft} \large
%\begin{tabular}{l}
%\today \hspace{8.4cm} Group 4 ``CASE"-HENK \\
%\end{tabular}
%\end{flushleft}
%
%\renewcommand{\arraystretch}{1.5}
%
%\end{titlepage}

\newgeometry{top=2.5cm, bottom = 3cm, left=2cm, right=2cm}

%\newpage
%
%\setcounter{tocdepth}{2}
%
%\tableofcontents
%\newpage

% {\_} underscore

%------------------------------------------------
\section{Karto filter code}

Understandig the Karto filter code
Refer for the theory to paper [1]

\section{Tuning KartoSLAM} \label{sec:Graphslam}

Parameters and definition for the KartoSLAM mapping approach. We are going to change and see the effect of the main parameters, as in GMapping (particles, linear and angular update, \textbf{no presence of resampling threshold}). \\
Strategy is starting with a low number of particles and increase them gradually, report results in a Table. \\
We select the best configuration as soon as the map presents an acceptable quality suitable for localization.
\textbf{IMP}: Before mapping with the graph mapper, mount the laser upside down.

\begin{center}
\captionof{table}{Mapper.yaml parameters description, default constant values}
\begin{tabular}{| m{12em} | m{13em}| m{12em}|} 
\hline
\textbf{Parameter} & \textbf{Value} & \textbf{Meaning} \\
\hline
grid{\_}resolution & 0.05 [m] &  grid map resolution, same as for GMapping \\
\hline
range{\_}threshold & 30 & Keep it for now at this constant value, but a suitable definition should be found.   \\
\hline
map{\_}update{\_}rate & 5 [s] & same for GMapping \\
\hline
min{\_}map{\_}size & 100 [m] &  initial map size [m], same as GMapping \\
\hline
\end{tabular}
\end{center}

The range threshold has an influence in the mapping quality, the higher the better the map using the same particles. I am looking for a definition for it. \\
The strategy for determining the optimal value is the following. We start with a very low number of particles (10,50) and progressively higher the threshold, till its increase has no more influence on the mapping process. An optimal of 90 has been found for now, given the particles of row 1 in Table 2. \\
What has been noted is the lower this \textbf{range threshold}, the lower is also the range in which the \textbf{grid mapping} is performed.


\begin{center}
\captionof{table}{Number of particles variations}
\begin{tabular}{| m{12em} | m{13em}|} 
\hline
\textbf{Particles number} & \textbf{Resulting map} \\
\hline
min=10, max=50, range=90 & \textbf{This is the optimal}. See the folder graphmapping-default to see all Figures resulted in the different mapping process. \\
\hline
\end{tabular}
\end{center}

This IS DONE !

\section{References}

[1]: Efficient Sparse Pose Adjustment for 2D Mapping, K.Konolige, G. Grisetti

\end{document}



% == TABLE ==
%begin{table}[h!]
 % \centering
  %\caption{Caption for the table.}
 % \label{tab:table1}
 % \begin{tabular}{ccc}
 %   \toprule
  %  Some & actual & content\\
   % \midrule
   % prettifies & the & content\\
   % as & well & as\\
  %  using & the & booktabs package\\
  %   \bottomrule
  %\end{tabular}
%\end{table}


% === ALGORITHM == 

\iffalse % multi-comment tool
\begin{algorithm}[!h]
   \caption{Kirsch, Rohig algorithm}
    \begin{algorithmic}[1]
    	\State $St-1 = St$
        \For{$i = 1$ to $N$} \Comment{With N the number of particles in the filter set by maxparticle parameter}
            \State $Spread $ $particles$ $in$ $the$ $anchorbox$ $with$ $equations$ $1)$ $and$ $2)$ $of$ $[3]$ \Comment{This step is called $Global$ $Localization$}
            
            \State $xt[n] = p(xt|xt-1,ut)$ \Comment{Motion update - sample the particles from the motion update of the robot and move forward to estimate the error model functions}
            
        	\State $wt[n] = p(dnanoLOC|si)*p(dlaser|si)$ \Comment{Measurement update - si are the particles set with i the i-th index}
        	\State $St = St + <xt,wt>$ \Comment{add the state and weight to the total state space}
        	
        	\State $Perform$ $resampling$
        \EndFor
    \State $Return$ $St$

\end{algorithmic}
\end{algorithm}
\fi


\iffalse

\begin{figure}[!htb]
    \centering
    \begin{minipage}{.5\textwidth}
        \centering
        \includegraphics[width=0.7\linewidth, height=0.2\textheight]{figures/amcl_param}
        \caption{The $amcl$ tunable parameters}
        \label{fig:amcl_param}
    \end{minipage}%
    \begin{minipage}{0.5\textwidth}
        \centering
        \includegraphics[width=0.7\linewidth, height=0.2\textheight]{figures/my_amcl_gmapping}
        \caption{Result of the Gmapping for the simple indoor environment}
        \label{fig:myamcl_map}
    \end{minipage}
 \end{figure}
 
 
 
 \begin{figure}[!htb]
	\center
	\includegraphics[width=1\textwidth]{figures/active_localization_node.png}
	\caption{An example of an active localization node}
	\label{fig:active_locnode}
\end{figure}


% underscore symbol {\_}


\fi